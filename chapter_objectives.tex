\chapter{Objectives}
\paragraph{}
The student will be asked to be on-site at the headquarters in Boston for a couple of weeks in order to meet the team and to get to know the company in order to gather all the information necessary to reach the objectives set further in this document. He will follow and join meetings to obtain a good insight in the requirements of the project and learn how to work under a Agile/Scrum based development methodology.
\paragraph{}
Being responsible for improving the Drupal Apache Solr search integration [1] project and the Acquia Search service is the common theme of the whole internship. This means adding additional features, keeping high quality and create upgrades/updates. The objective will be to exploit as much as possible from the latest Apache Solr 3.x branch while merging and keeping the software compatible with Apache Solr 1.4.

\paragraph{}
Communication will be a crucial part in order to succeed. The project has a worldwide scope, reaching out to more companies then just Acquia. This means he will have to be able to consult and make decisions after talking with a lot of end-users and other stakeholders. English will be the language of choice. This can happen by means of chat (IRC), on the Drupal community website [14] , giving presentations in conferences or taking interviews.
Finally the ability to work remotely, over a large distance and in a team, is an important skill to acquire.

In the beginning of the internship this was itemized in a list.
\begin{itemize}
  \item Bring Facet Api [2] for Drupal 7 to a stable Release Candidate.
  \item Create a multisite module to search between 2 or more Drupal sites.
  \item Update the Acquia Search service to the latest stable Apache Solr version. Upgrade the custom java code that was written to be able to authenticate customers.
  \item Backport to a new Drupal 6 branch all the new features that have been programmed into the Drupal 7 version of the Apache Solr Search Integration Module. This includes the backporting of the multisite module.
\item Achieve mastery of the agile/scrum process, the open source software engineering methods, and the team communication processes used by Acquia.
\item Empower the community to use the Apache Solr Search Integration project by means of Presentations, Blog posts and other interactions with community members.
\end{itemize}