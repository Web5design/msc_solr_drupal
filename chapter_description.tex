\chapter{Description and Terms}
\section{Acquia}
\paragraph{} 
Acquia is a commercial open source software company providing products, services, and technical support for the open source Drupal social publishing system and was founded by Dries Buytaert, the original creator and project lead of the Drupal project. With over two million downloads since inception, Drupal is used by web developers worldwide to build sophisticated community websites. Diverse organizations use Drupal as their core social publishing system for external facing websites and internal collaboration applications.

\paragraph{}
Acquia Search is a plug-and-play service within the Acquia Network [16], built on Apache Solr [4] and available for any Drupal 6 or Drupal 7 site. Acquia Search offers site visitors faceted search navigation and content recommendations to help them find valuable information faster. It is a fully redundant, high performance cloud service, with no software to install or servers to manage.

\section{Drupal}
Information about Drupal + History etc etc..

\section{Apache Solr}
Information about Apache Solr + History etc etc..

\section{Personal History}
My story with Drupal starts in the beginning of 2007. During my Bachelor I was asked, together with a group of people, to make a community site in Drupal to see what is was capable of. Now, almost 5 years later I still don't fully know what it is capable of since it keeps evolving and growing.

- Kaho
- Project (Reference?)
- Worked 1 year at Krimson (Reference, maybe recommendation?) where my interest in Solr and Search started and created my first real module (apachesolr\_ubercart for Drupal 6)

- Worked at Ateneatech (Officially using UPC contracts)
- Worked at AtSistemas as one of the reference engineers for a huge Solr and Drupal powered website. (http://www.elsevier.es)
During my studies at UPC I kept following the Drupal development and made numerous discussions with people and teachers on how software engineering should look at these projects. In the course of Advanced Web Technologies I presented Drupal a couple of times from different angles. (References)

There was only 1 logical step possible and that was doing my Master thesis with Acquia. During my Erasmus period in Portugal I attended a Drupal Camp (And even presented some Solr technologies, see reference) and I've met Robert Douglas, one of the creators of the Apache Solr Integration Project for Drupal and approached him with the question if I would be able to do my internship with Acquia. After a long process with the UPC and with Acquia everything was set and the pieces of the puzzle fell in place.