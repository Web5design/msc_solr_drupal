\chapter{Description}
\label{chap:description}
\section{Personal History}
\paragraph{}
My story with Drupal starts in the beginning of 2007. I've done my Bachelor degree at the Katholic University of Ghent\footnote{\url{http://www.kaho.be}}. During the second year of my Bachelor I was asked, together with a 2 other people, to make a community site in Drupal to see what is was capable of. This was created in Drupal 5 and while it wasn't as powerful as it is now we were already able to integrate LDAP into the website and customize it to our needs. I do have to admit that we, as a group, made numerous mistakes against the ethics of customizing Drupal back in the days. \footnote{Insert drupal code standards here} 

\paragraph{}
This Drupal 5 project ended, my bachelor ended and I started looking for a job and ended up with a small company called Krimson \footnote{\url{http://www.krimson.be}}. This company taught me the correct way of programming Drupal and Immediately they said : "You can start with Drupal 6, very new and way better compared with the previous version". And so I did, I started creating websites full of interactivity and community, backends that connect directly to databases running on a mainframe and even planted the initial bean of interest in search (Solr) that later would appear to grow out as this thesis topic. That website is still active on the address of \url{http://www.kortingsreus.nl}. It is also there that I created my first Drupal module, namely apachesolr\_ubercart \footnote{\url{http://drupal.org/project/apachesolr_ubercart}}
\paragraph{}
Afterwards I moved to Spain to study at the UPC \footnote{\url{http://www.upc.edu/}} and started to work half time at Ateneatech \footnote{\url{http://ateneatech.com/}} and later for AT-Sistemas as one of the reference engineers for a huge Solr and Drupal powered website. \footnote{\url{http://www.elsevier.es}}. Louis Toubes, one of the lead engineers, was able to give a small reference : "Nick tiene una capacidad innata de aprender por sí solo nuevas tecnologias y lo más importante es que el disfruta con ello. Sin duda, Nick es una de esas personas que desde el primer momento que la conoces sabes que aprenderás mucho de él."

\paragraph{}
During my studies at UPC I kept following the Drupal development and made numerous discussions with people and teachers on how software engineering should look at these projects. In the course of Advanced Web Technologies I even presented Drupal in classes : "Drupal as a framework" \footnote{\url{http://prezi.com/10_1ssdjroao/}}
\paragraph{}
There was only 1 logical step possible as my next step and that was doing an internship/thesis with Acquia. During my Erasmus period in Portugal I attended a Drupal Camp and I was also a presenter at the conference {\footnote{\url{http://lisboa2011.drupal-pt.org/sessoes/apachesolr-the-complete-search-solution}} and I've met Robert Douglas, one of the creators of the Apache Solr Integration Project for Drupal and approached him with the question if I would be able to do my internship with Acquia. After a long process with the UPC and with Acquia everything was set and the pieces of the puzzle fell in place.
\paragraph{}
Now,  being 2012 and a couple of years later I still don't fully know what Drupal and all its derivatives are capable of since it keeps evolving and growing. And that's good because it keeps me as a person growing and it keeps me up to date with most of the latest web technologies.

\section{Acquia}
\paragraph{} 
Acquia is a commercial open source software company providing products, services, and technical support for the open source Drupal social publishing system and was founded by Dries Buytaert, the original creator and project lead of the Drupal project. With over two million downloads since inception, Drupal is used by web developers worldwide to build sophisticated community websites. Diverse organizations use Drupal as their core social publishing system for external facing websites and internal collaboration applications.

\paragraph{}
Acquia Search\footnote{\url{http://acquia.com/productsservices/acquia-search}} is a plug-and-play service within the Acquia Network \footnote{\url{http://www.acquia.com/products-services/acquia-network}}, built on Apache Solr \footnote{\url{http://drupal.org/project/apachesolr}} and is available for any Drupal 6 or Drupal 7 site. Acquia Search offers site visitors faceted search navigation and content recommendations to help them find valuable information faster. It is a fully redundant, high performance cloud service, with no software to install or servers to manage.

\section{Drupal}
\paragraph{}
Drupal is originally written by Dries Buytaert as a message board. It became an open source project in 2001. Drupal is a free and open-source content management system (CMS) written in PHP and distributed under the GNU General Public License. It is used as a back-end system for at least 1.5\% of all websites worldwide ranging from personal blogs to corporate, political, and government sites including whitehouse.gov and data.gov.uk. It is also used for knowledge management and business collaboration.

\paragraph{}
The standard release of Drupal, known as Drupal core, contains basic features common to content management systems. These include user account registration and maintenance, menu management, RSS-feeds, page layout customization, system administrationand even a basic search functionalty. The Drupal core installation can be used as a brochureware website, a single- or multi-user blog, an Internet forum, or a community website providing for user-generated content.

\paragraph{}
There are more than 12,000 free community-contributed addons, known as contrib modules, available to alter and extend Drupal's core capabilities and add new features or customize Drupal's behavior and appearance. Because of this plug-in extensibility and modular design, Drupal is sometimes described as a content management framework. Drupal is also described as a web application framework, as it meets the generally accepted feature requirements for such frameworks.
While Drupal core (7) comes with advanced search capabilities it is still restricted by regular databases. \footnote{Any database that is compliant with the SQL standard should be able to run Drupal 7}

\section{Apache Solr}
\paragraph{}
Apache Solr is an open source enterprise search platform created on top of the Apache Lucene project. Its major features include powerful full-text search, hit highlighting, faceted search, dynamic clustering, database integration, and rich document (e.g., Word, PDF) handling. Providing distributed search and index replication, Solr is highly scalable.

\paragraph{}
Apache Solr is written in Java and runs as a standalone full-text search server within a servlet container such as Apache Tomcat. Solr uses the Lucene Java search library at its core for full-text indexing and search, and has REST-like HTTP/XML and JSON APIs that make it easy to use from virtually any programming language. 
\paragraph{}
Solr's powerful external configuration allows it to be tailored to almost any type of application without Java coding, and it has an extensive plugin architecture when more advanced customization is required. Further in this work you can find an example of such a plugin written to provide extra functionality for Acquia.
\paragraph{}
When looking at a Lucene index, compared to a relational database, it seems that the index is one database table and has very fast lookups and different specific filters for text search. It takes time to create an index like this. Solr adds a front-end to the Lucene backend and many other additions.
\paragraph{}
Fundamentally, Solr is very simple. One feeds it with information and afterwards you query Solr and receive the information you requested. Solr allows applications to build indexes based on different fields\footnote{Fields are different kinds of entries}. These fields are defined in a schema which tells Solr how it should build the index.
