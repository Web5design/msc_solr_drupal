\chapter{Analysis}
This chapter describes the process of analyzing what existed in the apachesolr module when I started and points out some of the improvements that should be made. It also takes a deeper look in the Facet Api \footnote{\url{http://www.drupal.org/project/facetapi}}module to see how it is structured. 
Apache Solr was benchmarked with different configurations to find out the most optimal ones. 

\section{Apache Solr}
\paragraph{Version conflicts}
Apache Solr exists out of a couple parts. The analysis part tries to explain you all it entails. When Drupal 6 came out and became popular, there was only one version of Apache Solr available. This was version 1.4 and was not yet merged with the Lucene branch. Solr's version number was synced with Lucene following the Lucene/Solr merge, so Solr 3.1 contains Lucene 3.1. Solr 3.1 is the first release after Solr 1.4.1. All the explanation that follows will be for Solr 3.x since this is the version that is used during the creation of this project.

\paragraph{Fields}
There are different field types defined in the original schema.xml that used to come with the module. A field type has four types of information.
\begin{itemize}
  \item The name of the field type
  \item An implementation class name
  \item If the field type is TextField, a description of the field analysis for the field type
  \item Field attributes
\end{itemize}
To illustrate this there is listing \ref{code:fieldtypedefinition} as an example of a field type definition as it is used in the schema provided with the Apache Solr Module and also a list of all possible field types in listing \ref{lst:fieldtypedefinition}

\begin{longtable}{ | 1 | p{11cm} |}
    \hline
    Class & Description \\ \hline
    BCDIntField & Binary-coded decimal (BCD) integer. BCD is a relatively inefficient 
    encoding that offers the benefits of quick decimal calculations and quick conversion to a string. \\ \hline
    BCDLongField & BCD long integer \\ \hline
    BCDStrField & BCD string \\ \hline
    *BinaryField & Binary data \\ \hline
    *BoolField & Contains either true or false. Values of "1", "t", or "T" in the first character are interpreted as true. Any other values in the first character are interpreted as false. \\ \hline
    ByteField & Contains an array of bytes. \\ \hline
    *DateField & Represents a point in time with millisecond precision. \\ \hline
    DoubleField & Double (64-bit IEEE floating point) \\ \hline
    ExternalFileField & Pulls values from a file on disk. See the section below on working with external files.\\ \hline
    FloatField & Floating point (32-bit IEEE floating point) \\ \hline
    IntField & Integer (32-bit signed integer) \\ \hline
    LongField & Long integer (64-bit signed integer) \\ \hline
    Point &  \\ \hline
    *RandomSortField & Does not contain a value. Queries that sort on this field type will return results in random order. Use a dynamic field to use this feature. \\ \hline
    ShortField & Short integer \\ \hline
    SortableDoubleField & The Sortable* fields provide correct numeric sorting. If you use the plain types (DoubleField, IntField, and so on) sorting will be lexicographical instead of numeric. \\ \hline
    SortableFloatField & Numerically sorted floating point \\ \hline
    SortableIntField & Numerically sorted integer \\ \hline
    SortableLongField & Numerically sorted long integer \\ \hline
    *StrField & String (UTF-8 encoded string or Unicode) \\ \hline
    *TextField & Text, usually multiple words or tokens \\ \hline
    *TrieDateField & Date field accessible for Lucene TrieRange processing \\ \hline
    *TrieDoubleField & Double field accessible Lucene TrieRange processing \\ \hline
    TrieField & If this type is used, a "type" attribute must also be specified, with a value of either: integer, long, float, double, date. Using this field is the same as using any of the Trie*Fields. \\ \hline
    *TrieFloatField & Floating point field accessible Lucene TrieRange processing \\ \hline
    *TrieIntField & Int field accessible Lucene TrieRange processing \\ \hline
    *TrieLongField & Long field accessible Lucene TrieRange processing \\ \hline
    *PointType & For spatial search: An arbitrary n-dimensional point, useful for searching sources such as blueprints or CAD drawings. \\ \hline
    *LatLonType &  Latitude/Longitude as a 2 dimensional point. Latitude is always specified first.\\ \hline
    *GeoHashField & Representing a Geohash\footnote{Geohash is a defined standard. More on wikipedia : \url{http://en.wikipedia.org/wiki/Geohash}} field. The field is provided as a lat/lon pair and is internally represented as a string.\\ \hline
    UUIDField & Universally Unique Identifier (UUID). Pass in a value of "NEW" and Solr will create a new UUID. \\ 
     \hline
\end{longtable}
\captionof{listing}{All field type definitions. Marked with a star are the ones that are not used in the Apache Solr Search Integration module for Drupal  \label{lst:fieldtypedefinition}}

\paragraph{}
More information on how to configure these fields can be found on \\
\url{http://lucidworks.lucidimagination.com/display/solr/Solr+Field+Types}

\newpage
\inputminted[fontsize=\scriptsize,linenos]{xml}{code_examples/schema_fieldtype.xml}
\captionof{listing}{Example of a text field type definition \label{code:fieldtypedefinition}}


\section{Apachesolr module for Drupal 6 version 6.x-2.x}
\section{Apachesolr module for Drupal 7 version 7.x-1.x-beta5}
\section{Facetapi module for Drupal 7 version 7.x-1.x}
\section{Acquia Search for Drupal 6 and 7}
