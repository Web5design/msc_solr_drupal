%  Overzichtsbladzijde met samenvatting

\newpage
{
\setlength{\baselineskip}{14pt}
\setlength{\parindent}{0pt}
\setlength{\parskip}{8pt}

\begin{center}

\noindent \textbf{\huge Improving Acquia search and the Apache Solr Module\\[8pt]}


by 

Nick VEENHOF

Master Thesis in order to acquire a Master in Information Technology

Academic Year 2011--2012

Supervisor: B.~Chris BROOKINS\\
Co-Supervisor: Dr.~Ir.~Peter WOLANIN\\
Tutor/Professor: Prof.~Dr.~Ir.~Carles FARR\'{E} TOST\\

Barcelona School of Informatics (FIB)\\
Master in Information Technology\\
BarcelonaTech

\end{center}

\section*{Abstract}

% TODO: Abstract

This work is intended to show the upgrade process of a module on drupal.org using community tools. This study was performed as part of an internship at Acquia Inc during a period of 5 months. More specifically it was focussed on creating a stable release of the Apache Solr Search Integration Module for Drupal 7 and eventually also backport this to Drupal 6. 
Firstly there was  an analysis state and a brief introduction to how the system worked and how to co-operate with an existing Open Source community. From this, existing problems were identified and thrown in a roadmap. Community projects have a very dynamic rhythm and issues could rise up or get resolved because thousands of persons had access to the code base. These challenges are described and tips are given on how to cope with such a dynamic development process.
This study also describes the challenges of a backport and how to resolve them. 
Finally, there is an explanation of the Acquia Search service and the process to upgrade the server park from Solr 1.4 to Solr 3.x (initially 3.4, finally 3.5) that includes the process of writing a Java servlet for managing authentication over rest services using RFC2104 HMAC encryption.

\section*{Keywords}
Drupal, Apache Solr, Lucene, Acquia, Acquia Search, Acquia Network, Search Technology
}

\newpage % Necessary to prevent a header to pop up on this page
