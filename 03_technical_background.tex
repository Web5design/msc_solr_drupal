\chapter{Description}
\label{chap:description}
This chapter gives a short overview of technical concepts used in this work. It is not intended
to be exhaustive and references are given for further reading

\section{Acquia}
\paragraph{} 
Acquia is a commercial open source software company providing products, services, and technical support for the open source Drupal social publishing system and was founded by Dries Buytaert, the original creator and project lead of the Drupal project. With over two million downloads since inception, Drupal is used by web developers worldwide to build sophisticated community websites. Diverse organizations use Drupal as their core social publishing system for external facing websites and internal collaboration applications.

\paragraph{}
Acquia Search\footnote{\url{http://acquia.com/products-services/acquia-search}} is a plug-and-play service within the Acquia Network \footnote{\url{http://www.acquia.com/products-services/acquia-network}}, built on Apache Solr \footnote{\url{http://drupal.org/project/apachesolr}} and is available for any Drupal 6 or Drupal 7 site. Acquia Search offers site visitors faceted search navigation and content recommendations to help them find valuable information faster. It is a fully redundant, high performance cloud service, with no software to install or servers to manage.

\section{Apache Solr}
\paragraph{}
Apache Solr is an open source enterprise search platform created on top of the Apache Lucene project. Its major features include powerful full-text search, hit highlighting, faceted search, dynamic clustering, database integration, and rich document (e.g., Word, PDF) handling. Providing distributed search and index replication, Solr is highly scalable.

\paragraph{}
Apache Solr is written in Java and runs as a standalone full-text search server within a servlet container such as Apache Tomcat. Solr uses the Lucene Java search library at its core for full-text indexing and search, and has REST-like HTTP/XML and JSON APIs that make it easy to use from virtually any programming language. 
\paragraph{}
Solr's powerful external configuration allows it to be tailored to almost any type of application without Java coding, and it has an extensive plugin architecture when more advanced customization is required. Further in this work you can find an example of such a plugin written to provide extra functionality for Acquia.
\paragraph{}
When looking at a Lucene index, compared to a relational database, it seems that the index is one database table and has very fast lookups and different specific filters for text search. It takes time to create an index like this. Solr adds a front-end to the Lucene backend and many other additions.
\paragraph{}
Fundamentally, Solr is very simple. One feeds it with information (documents) and afterwards you query Solr and receive the documents that match the query. Solr allows applications to build indexes based on different fields\footnote{Fields are different kinds of entries}. These fields are defined in a schema which tells Solr how it should build the index.

Using analyzers and tokenizers a search query is processed. Field analyzers are used both during ingestion, when a document is indexed, and at query time. An analyzer examines the text of fields and generates a token stream. Analyzers may be a single class or they may be composed of a series of tokenizer and filter classes.

Tokenizers break field data into lexical units, or tokens. Filters examine a stream of tokens and keep them, transform or discard them, or create new ones. Tokenizers and filters may be combined to form pipelines, or chains, where the output of one is input to the next. Such a sequence of tokenizers and filters is called an analyzer and the resulting output of an analyzer is used to match query results or build indices.

Although the analysis process is used for both indexing and querying, the same analysis process need not be used for both operations. For indexing, you often want to simplify, or normalize, words. For example, setting all letters to lowercase, eliminating punctuation and accents, mapping words to their stems, and so on. Doing so can increase recall because, for example, "ram", "Ram" and "RAM" would all match a query for "ram". To increase query-time precision, a filter could be employed to narrow the matches by, for example, ignoring all-cap acronyms if you're interested in male sheep, but not Random Access Memory.

The tokens output by the analysis process define the values, or terms, of that field and are used either to build an index of those terms when a new document is added, or to identify which documents contain the terms your are querying for.

\section{Drupal}
\paragraph{History}
Drupal is originally written by Dries Buytaert as a message board. It became an open source project in 2001. Drupal is a free and open-source content management system (CMS) written in PHP and distributed under the GNU General Public License. It is used as a back-end system for at least 1.5\% of all websites worldwide ranging from personal blogs to corporate, political, and government sites including whitehouse.gov and data.gov.uk. It is also used for knowledge management and business collaboration.

\paragraph{}
The standard release of Drupal, known as Drupal core, contains basic features common to content management systems. These include user account registration and maintenance, menu management, RSS-feeds, page layout customization, system administrationand even a basic search functionalty. The Drupal core installation can be used as a brochureware website, a single- or multi-user blog, an Internet forum, or a community website providing for user-generated content.

\paragraph{Basic understanding}
A single web site could contain many types of content, such as informational pages, news items, polls, blog posts, real estate listings, etc. In Drupal, each item of content is called a node (internally called an entity), and each node belongs to a single content type (internally called entity type), which defines various default settings for nodes of that type, such as whether the node is published automatically and whether comments are permitted. (Note that in versions below 7 of Drupal, content types were known as node types.)

\paragraph{Contributed Modules}
There are more than 12,000 free community-contributed addons, known as contrib modules, available to alter and extend Drupal's core capabilities and add new features or customize Drupal's behavior and appearance. Because of this plug-in extensibility and modular design, Drupal is sometimes described as a content management framework. Drupal is also described as a web application framework, as it meets the generally accepted feature requirements for such frameworks.
While Drupal core (7) comes with advanced search capabilities it is still restricted by regular databases. \footnote{Any database that is compliant with the SQL standard should be able to run Drupal 7}. The module that was created during this work is also defined as a contributed module.

\paragraph{Apache Solr Search Integration}
The Drupal module integrates Drupal with the Apache Solr search platform. Faceted search is supported if the facet API module is used. Facets will be available for you ranging from content author to taxonomy to arbitrary fields. The module also includes functionalities such as :
\begin{itemize}
  \item Search pages, eg.: multiple search pages with optionally customized search results.
  \item Multiple environments to support multiple Solr servers.
  \item Comes with support for the node content type including dynamic fields.
  \item Can override the taxonomy pages and use output from Solr to generate taxonomy overview pages.
  \item Can override the user content listing pages using output from Solr to generate these.
  \item Custom Content types (entities) indexing through hooks.
  \item Add biases and boosts to specific fields or content types
  \item Range Query type, that in combination with facet API and Facet Api Slider a very rich faceting experience delivers to the end user.
  \item Supports a lot of customizations without having to modify the source code
\end{itemize}



