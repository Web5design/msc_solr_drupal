\chapter{Implementation}

I started my internship September 22nd in Boston. While being on-site and I learned how Acquia manages processes and works together with the community in order to reach business goals. Also watching them work with a lot of remote employees was already a very valuable lesson. More about this experience can be read in the article [9] I wrote about this. In addition to this I have helped with the creation of some new modules such as Facet Slider [12] and Apachesolr sort UI [13].

As for addressing the public on this subject, I have recently given a presentation ‘Drupal Search’ [6] explaining to more than 60 attendees what was done with the module and where it was heading to in Belgium. Lots of open communication [10] has happened within the community in the Apache Solr Issue Queue. [11] In total I have given 4 presentations with a combined total of more then 200 attendants. Since my involvement with the project there are about 2500 websites using the Drupal 7 version of the module and about 10000 users in total using the module for Drupal 6 and Drupal 7 combined.
  
I’ve contributed several blog posts about this topic and this internship
\begin{itemize}
  \item Presentation about Drupal Search for the DUG group in November 2011. \cite{url10}
  \item Simple guide to install Apache Solr 3.x for Drupal 7 on a unix machine [7]
  \item Adding a custom plugin to the Apache Solr Project [8] 
  \item A Story of an intern at Acquia [9] 
\end{itemize}

Objectives 1, 2 and 3 are in progress and are nearing its completion status. Objective 4, updating the Acquia Search service to the latest stable Apache Solr version, made great progress and is currently being tested by a client of Acquia that required this change. 
Every day there is a daily call with Peter Wolanin to keep the daily objectives clear and to clear out any issues that could block progress.

\section{Apachesolr module for Drupal 6 version 6.x-3.x}
\section{Apachesolr module for Drupal 7 version 7.x-1.0}
\section{Functional requirements}
\subsection{Search Environments}
\subsection{Search pages}
\subsection{Query Object}
\subsection{Apaches Solr Document}
\subsection{Entity layer}
\section{Non-Functional Requirements}
\subsection{User interface}
\subsection{Usability}
\subsection{Performance}
\subsection{Security}
\subsection{Legal}